\documentclass{tufte-handout}

\usepackage{xcolor}

% set hyperlink attributes
\hypersetup{colorlinks}

% set image attributes:
\usepackage{graphicx}
\graphicspath{ {images/} }

% indent subsections
\newenvironment{subs}
  {\adjustwidth{3em}{0pt}}
  {\endadjustwidth}

% ============================================================

% define the title
\title{SOC 4015/5050: Lecture 12 Functions}
\author{Christopher Prener, Ph.D.}
\date{Fall 2018}
% ============================================================
\begin{document}
% ============================================================
\maketitle % generates the title
% ============================================================

\vspace{5mm}
\section{Packages}
\begin{itemize}
\item \texttt{ggplot2}
\item \texttt{ggthemes}
\item \texttt{RColorBrewer}
\item \texttt{stats}
\end{itemize}

\vspace{5mm}
\section{Creating Plots for Dissemination}
\begin{subs}
\subsection{Increase Geom Size}
\noindent \texttt{ggplot2::}\texttt{geom(size = \textit{val})}

\vspace{3mm}
\subsection{Add Stroke to Scatter Plot Points}
\begin{fullwidth}
\noindent \texttt{ggplot2::}{\color{red}\texttt{geom\_point}}\texttt{(mapping = aes(fill = \textit{var}), pch = 21)}
\end{fullwidth}

\vspace{3mm}
\subsection{Increase Font Size on Entire Plot}
\noindent \texttt{ggplot2::}{\color{red}\texttt{theme\_grey}}\texttt{(base\_size = \textit{val})}

\vspace{3mm}
\subsection{Alternative Themes}
\noindent The \texttt{ggthemes} package contains 14 alternative themes, see their \href{https://cran.r-project.org/web/packages/ggthemes/vignettes/ggthemes.html}{introductory vignette} for a list. Each theme has its own function (e.g. \texttt{ggthemes::}{\color{red}\texttt{theme\_hc}}\texttt{()}).

\newpage
\subsection{Labels Function}
\noindent \texttt{ggplot2::}{\color{red}\texttt{labs}}\texttt{(}
\par \texttt{title = "plot title",}
\par \texttt{subtitle = "plot subtitle",}
\par \texttt{caption = "caption text",}
\par \texttt{x = "x-axis label",}
\par \texttt{y = "y-axis label"}
\par \noindent \texttt{)}

\vspace{3mm}
\subsection{Adjust Legend Sizing}
\noindent \texttt{ggplot2::}{\color{red}\texttt{theme}}\texttt{(legend.key.size = unit(\textit{val}, units = "cm"))}\sidenote{\texttt{ggplot2} accepts inches and millimeters as units of measure in addition to centimeters.}

\vspace{3mm}
\subsection{Legend Title and Labels}
\noindent \texttt{ggplot2::}{\color{red}\texttt{scale\_fill\_discrete}}\texttt{(labels = c("label1", \\"label2"), name = "\textit{legend title}")}\sidenote{If the \texttt{geom} uses the \texttt{color} argument and not \texttt{fill}, this would need to be \texttt{ggplot2::scale\_color\_discrete()}.}
\end{subs}

\vspace{5mm}
\section{Color Brewer}
\begin{subs}
\subsection{Display All Palettes}
\noindent \texttt{RColorBrewer::}{\color{red}\texttt{display.brewer.all}}\texttt{()}

\vspace{3mm}
\subsection{Create Vector of Hex Values from Palette}
\noindent \texttt{RColorBrewer::}{\color{red}\texttt{brewer.pal}}\texttt{(\textit{n}, "\textit{name}")}

\vspace{3mm}
\subsection{Add Discrete Color Palette}
\noindent \texttt{ggplot2::}{\color{red}\texttt{scale\_fill\_brewer}}\texttt{(palette = "\textit{name}", \\labels = c("label1", "label2"), \\name = "\textit{legend title}")}
\end{subs}

\vspace{5mm}
\section{Linear Model}
\noindent \texttt{stats::}{\color{red}\texttt{lm}}\texttt{(formula = \textit{yvar} \textasciitilde\ \textit{xvar}, data = \textit{dataFrame})}
% ============================================================
\end{document}